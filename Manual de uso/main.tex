\documentclass{article}
\usepackage[utf8]{inputenc}
\usepackage[spanish]{babel}
\usepackage{listings}
\usepackage{graphicx}
\graphicspath{ {images/} }
\usepackage{cite}

\begin{document}

\begin{titlepage}
    \begin{center}
        \vspace*{1cm}
            
        \Huge
        \textbf{Parcial II}
            
        \vspace{0.5cm}
        \LARGE
        Manual de uso
            
        \vspace{1.5cm}
            
        \textbf{Juan Sebastian Anaya Regino}
            
        \vspace{0.9cm}
        \centering
        \includegraphics[width=6cm]{images/logo.png}
            
        \vfill
            
        \vspace{0.8cm}
            
        \Large
        Despartamento de Ingeniería Electrónica y Telecomunicaciones\\
        Universidad de Antioquia\\
        Medellín\\
        Septiembre de 2021
            
    \end{center}
\end{titlepage}

\section{¿Como lo uso?}
Para usar correctamente el programa debe tener en cuenta las siguientes instrucciones: \\ \\
1. Asegúrese de que la imagen que desea procesar se encuentra dentro de la carpeta 'images' que está en el mismo directorio del proyecto. \\ \\
2. El formato de la imagen debe ser '.jpg'. \\ \\
3. Cuando el programa le pida que ingrese el nombre de la imagen asegúrese de escribir únicamente el nombre de la imagen, abstenerse de enviar el nombre de la imagen con su respectiva extensión. \\ \\
4. El programa creará un archivo '.txt' dentro de la carpeta 'arduino’ que está en el mismo directorio del proyecto, este archivo tendrá por nombre el nombre de la imagen que se le dio al programa junto con las dimensiones de esta después de procesarla. Ejemplo, si al programa se le entrega una imagen de nombre 'bandera.jpg', el archivo que se creará en 'arduino' tendrá por nombre 'bandera12x12.txt'.\\ \\
5. Copie el contenido del archivo '.txt' correspondiente a la imagen que envió, luego diríjase al siguiente link 'https://www.tinkercad.com/things/bKXG30GX9e9' y abra la pestaña código, pegue el contenido antes copiado justo en parte que se le indica. \\

Si realizó correctamente los pasos anteriores podrá visualizar la imagen que invio al programa en una matriz de leds de 12x12.

\end{document}