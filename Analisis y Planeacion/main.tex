\documentclass{article}
\usepackage[utf8]{inputenc}
\usepackage[spanish]{babel}
\usepackage{listings}
\usepackage{graphicx}
\graphicspath{ {images/} }
\usepackage{cite}

\begin{document}

\begin{titlepage}
    \begin{center}
        \vspace*{1cm}
            
        \Huge
        \textbf{Parcial II}
            
        \vspace{0.5cm}
        \LARGE
        Análisis y planeación
            
        \vspace{1.5cm}
            
        \textbf{Juan Sebastian Anaya Regino}
            
        \vspace{0.9cm}
        \centering
        \includegraphics[width=6cm]{images/logo.png}
            
        \vfill
            
        \vspace{0.8cm}
            
        \Large
        Despartamento de Ingeniería Electrónica y Telecomunicaciones\\
        Universidad de Antioquia\\
        Medellín\\
        Septiembre de 2021
            
    \end{center}
\end{titlepage}

\section{Análisis}
El parcial plantea un manejo de imágenes, las cuales deberán ser mostradas en una matriz de leds(en este caso será una matriz de leds de 8x8), sin embargo el mayor problema reside en solucionar el problema de redimensionar una imagen de tal forma que esta pueda representarse en la matriz de leds de 8x8, y que aún pueda distinguirse. \\newline
Para realizar el redimensionamiento de la imagen será necesario crear métodos que se encarguen de dicha función, a continuación se muestra un análisis que permitirá elaborar dichos métodos:

\subsection{Submuestreo}
Para el problema en que la imagen sea más grande que la matriz de leds se deberá realizar el proceso de sobremuestreo, para este se ha hecho el siguiente análisis:
Las dimensiones de una imagen están dadas en forma de Ancho x Alto FIGURA1, entonces para hacer el proceso de submuestreo se tomará el ancho de la imagen y se dividirá por el número de leds que corresponden al ancho de la matriz, de igual forma se hará con el alto.


\end{document}
